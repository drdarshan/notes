\documentclass[12pt]{article}
\usepackage{fullpage}
\usepackage{amsmath}
\usepackage{amssymb}
\usepackage{tikz}
\usepackage{algorithm2e}
\usepackage{bm}
\setlength{\parindent}{0pt}
\newcommand{\defeq}{\ensuremath\stackrel{\text{\tiny def}}{=}}
\newcommand{\trans}[1]{\ensuremath{#1}^{\mathsf{T}}}
\DeclareMathOperator*{\argmin}{arg\,min} % thin space, limits underneath in displays
\title{Machine Learning Notes}
\author{R. Sudarshan}
\date{\today}
\begin{document}
\maketitle
\tableofcontents

\section{Introduction}
This is an evolving set of notes kept for my understanding of machine
learning algorithms.

\section{Vector calculus primer}
A lot of machine learning algorithms work on minimizing an objective
function $f: \mathbb{R}^{N}\rightarrow\mathbb{R}$. Typically, these
are done using algorithms like \textit{gradient descent} or
\textit{L-BFGS}. To understand machine learning algorithms, it is
therefore crucial to understand what gradients and Hessians are and
how they are computed.


So what exactly is the gradient of a function anyway? Given a function $f:\mathbb{R}^N\rightarrow\mathbb{R}$, we define
\[
\nabla{f}(w):\mathbb{R}\to\mathbb{R}^N \defeq \begin{bmatrix}\
  \frac{\partial{f}}{\partial{w_1}} \\
  \frac{\partial{f}}{\partial{w_2}} \\
  \vdots \\
  \frac{\partial{f}}{\partial{w_{N}}} \\
  \end{bmatrix}
\]
where in turn
\[
\frac{\partial{f}}{\partial{w_i}} = \lim_{h\to0}\frac{f(w+h\hat{e}_i) - f(w)}{h}
\]
and $\hat{e}_i$ is the unit-vector in the $i$-th dimension.

Gradients are really useful (at least in a optimization/machine
learning setting) when used in conjunction with the Taylor series
approximation for $f$ that I'm already comfortable with from 6.336. We have:
\[
f(w + \varepsilon) = f(w) + \varepsilon^{\mathsf{T}} \nabla{f}(w) + \text{h.o.t}
\]
Note that $\varepsilon \in \mathbb{R}^N$ is an ``epsilon''
displacement of $w$ in $N$-dimensional space. So as a first-order
approximation:
\[
\varepsilon^{\mathsf{T}}\nabla{f}(w) \approx f(w+\varepsilon) - f(w)
\]

\subsection{Gradient descent}
For a function $f:\mathbb{R}\to\mathbb{R}$, we know $f(w')$ is a
(local) maximum or minimum if
$\frac{\text{d}f}{\text{d}w}_{w'}=0$. But what about for functions
that are $\mathbb{R}^N\to\mathbb{R}$? We require the partial
derivative along \textit{each} axis $i = 1,2,\ldots,N$ to be zero, or in other words:
\[
\left.\nabla{f}\right|_{w'} = 0
\]

The \textit{gradient descent} algorithm is an iterative approach to
minimizing or maximizing $f$. Starting from a random guess, it tries
to find a point $w_{\text{opt}}$ such that
$\nabla{f}|_{w_{\text{opt}}} = 0$ by moving \textit{one step} in a
line that's in the opposite direction to the gradient at the current
guess, see the figure below:

\begin{figure}[htbp]
  \begin{center}
    \usetikzlibrary{shadows}
\begin{tikzpicture}
%%\draw [very thin, step=1cm] (-5,-5.5) grid (5.5,4);
\node (v1) at (-4,-3) {};
\node (v11) at (-5, -2) {};
\node (v2) at (5,-2) {};
\draw [very thick, -latex](v11) -- (v2);
\node (v3) at (-4,4) {};
\draw [very thick, -latex] (v1) edge (v3);
\node (v4) at (-3,3) {};
\node (v5) at (1,-1) {};
\node (v6) at (5,3) {};
\draw plot[smooth, tension=.7] coordinates {(v4) (v5) (v6)};
\node [shape=circle] (v14) at (4,1.5) {};
\node (v15) at (4,-2.5) {$\mathbf{w_0}$};
\draw [dashed] (4,-2) -- (4,1.5);
\node [shape=circle,  minimum size=0.02cm,  circular drop shadow,  text=black,  very thick,  draw=black!55,  top color=white, bottom color=gray!80, align=center] (v16) at (4,1.5) {};
\node (v18) at (5, 2.75) {$\mathbf{\frac{\partial f}{\partial w}}$};
\draw [red,thick,->](v16) -- (v18);
\node (v19) at (3,-2.5) {$\mathbf{w_1}$};
\node [shape=circle,  minimum size=0.02cm,  circular drop shadow,  text=black,  very thick,  draw=black!55,  top color=white, bottom color=gray!80, align=center] (v20) at (3,.3) {};
\end{tikzpicture}

  \end{center}
  \caption{One step of gradient descent algorithm.}
\end{figure}

The algorithm be written in pseudo-code as:

\begin{algorithm}[H]
  $w_0 \longleftarrow \texttt{rand}(N,1)$\;
  \While{not converged} {
    $w_{n+1} = w_{n} - \left.\nabla{f}\right|_{w_n}$\;
  }
\end{algorithm}

\section{Linear regression}

In linear regression, we are given a set of features or observations
$\mathbf{A} \in \mathbb{R}^{M\times N}$ along with a set of labels
$y\in \mathbb{R}^M$. We are then required to compute a set of weights
$w\in\mathbb{R}^N$ such that:
\[
w = \argmin_{u \in \mathbb{R}^M} \left\|\mathbf{A} u - y\right\|^2_2
\]

\subsection{Interlude into link functions and loss functions}

Given an observation $X \in \mathbb{R}^N$, the \textit{prediction} of
the model $\hat{y}$ can be obtained as:
\[
\hat{y} = f_{\text{Link}}\left(X^{\mathsf{T}}\ w\right)
\]
Here, $f_{\text{Link}}:\mathbb{R}\to\mathbb{R}$ is known as the
\textit{link function}. For linear regression, the link function is
just the identity, $f(x) = x$ but for logistic regression, the link function is the \textit{sigmoid}:
\[
\sigma(x) = \frac{1}{1 + e^{-x}}
\]

To compute the weights $w$, we are minimizing an objective function, referred to as the \textit{loss function} in literature:
\[
f_{\text{Loss}} = \sum\limits_{i=1}^{M} \left(\mathbf{A}[i,:]^{\mathsf{T}}\ u - y[i]\right)^2 = \left\|\mathbf{A} u - y\right\|^2_2
\]

So to summarize, the link function is used to generate predictions
$\hat{y}$ given a model whereas a loss function is actually used to
generate the parameters for the model (in the case of linear
regression, the weights $w$) given a training dataset.

\subsection{Gradients for linear regression}

So how exactly do you compute the weights for linear regression using
gradient descent? Obviously, we start by computing the gradient of the loss function:
\begin{equation}
  \label{eq:linreg}
  f(u) = \left\|\mathbf{A} u - y\right\|^2_2
\end{equation}

In many books such as ESL, the formula for the gradient is just
provided without any explanation. In others, the derivation is
unfortunately somewhat unclear. I want to provide what I hope is a
clear exposition from first principles.

Recall that the $i$-th component of the gradient is given by:
\[
(\nabla f(u))_i = \hat{e}^{\mathsf{T}}_i \nabla f(u) = \lim_{h\to0}\frac{f(u+h\hat{e}_i) - f(u)}{h}
\]
So we first compute $f(u+\varepsilon) - f(u)$ for some vector
$\varepsilon$ and set $\varepsilon$ to $h\hat{e}_i$ for
$i=1,2,\ldots,N$ to determine the gradient. We therefore have for the loss function \eqref{eq:linreg}:
\[
\begin{aligned}
  f(u + \varepsilon) - f(u) &=  \left\|\mathbf{A} (u + \varepsilon) - y\right\|^2_2 - \left\|\mathbf{A} u - y\right\|^2_2\\
  &= \left(\mathbf{A} (u + \varepsilon) - y\right)^{\mathsf{T}} \left(\mathbf{A} (u + \varepsilon) - y\right) - \left(\mathbf{A} u - y\right)^{\mathsf{T}}  \left(\mathbf{A} u - y\right) \\
  &= \left\{\left(\trans{(u+\varepsilon)}\trans{\mathbf{A}}-\trans{y}\right)\left(\mathbf{A}(u+\varepsilon)-y\right)\right\} - \left\{\left(\trans{u}\trans{\mathbf{A}}-\trans{y}\right)\left(\mathbf{A}u - y\right)\right\} \\
  &= \trans{\varepsilon}\mathbf{A}\varepsilon + 2\trans{\varepsilon}\trans{\mathbf{A}}\mathbf{A}u - 2\trans{\varepsilon}\mathbf{A}y \\
\end{aligned}
\]
Substituting $\varepsilon = h\hat{e}_i$ we get
\[
\frac{f(u+h\hat{e}_i)-f(u)}{h} = h\trans{\hat{e_i}}\trans{\mathbf{A}}\mathbf{A}\hat{e}_i + 2 \trans{\hat{e}_i}\left(\trans{\mathbf{A}}\mathbf{A}u - \mathbf{A}y\right)
\]
and therefore in the limit as $h\to0$:
\[
\trans{\hat{e}_i}\nabla{f}(u) = \trans{\hat{e}_i}\trans{\mathbf{A}}\left(\mathbf{A}u - y\right),\ i=1,2,\ldots, N
\]
or in other words
\begin{equation}
\nabla{f}(u) = \trans{\mathbf{A}}\left(\mathbf{A}u - y\right)
\end{equation}

\section{Perceptron learning}

\section{Logistic regression}
\subsection{Derivative of the sigmoid function}

\subsection{Multi-class logistic regression}

\subsection{Equivalence between sigmoid and softmax loss functions}
\section{Support vector machines}
\subsection{Online algorithms via PEGASOS}
\section{Evaluation}
\subsection{RoC curves and area under the curve}
\end{document}
