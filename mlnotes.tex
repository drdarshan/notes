\documentclass[12pt]{article}
\usepackage{fullpage}
\usepackage{amsmath}
\usepackage{amssymb}
\usepackage{tikz}
\newcommand{\defeq}{\ensuremath\stackrel{\text{\tiny def}}{=}}
\title{Machine Learning Notes}
\author{R. Sudarshan}
\date{\today}
\begin{document}
\maketitle
\tableofcontents

\section{Introduction}
This is an evolving set of notes kept for my understanding of machine
learning algorithms.

\section{Vector calculus primer}
A lot of machine learning algorithms work on minimizing an objective
function $f: \mathbb{R}^{N}\rightarrow\mathbb{R}$. Typically, these
are done using algorithms like \textit{gradient descent} or
\textit{L-BFGS}. To understand machine learning algorithms, it is
therefore crucial to understand what gradients and Hessians are and
how they are computed.


So what exactly is the gradient of a function anyway? Given a function $f:\mathbb{R}^N\rightarrow\mathbb{R}$, we define
\[
\nabla{f}(w):\mathbb{R}\to\mathbb{R}^N \defeq \begin{bmatrix}\
  \frac{\partial{f}}{\partial{w_1}} \\
  \frac{\partial{f}}{\partial{w_2}} \\
  \vdots \\
  \frac{\partial{f}}{\partial{w_{N}}} \\
  \end{bmatrix}
\]
where in turn
\[
\frac{\partial{f}}{\partial{w_i}} = \lim_{h\to0}\frac{f(w+h\hat{e}_i) - f(w)}{h}
\]
and $\hat{e}_i$ is the unit-vector in the $i$-th dimension.

Gradients are really useful (at least in a optimization/machine
learning setting) when used in conjunction with the Taylor series
approximation for $f$ that I'm already comfortable with from 6.336. We have:
\[
f(w + \varepsilon) = f(w) + \varepsilon^{\mathsf{T}} \nabla{f}(w) + \text{h.o.t}
\]
Note that $\varepsilon \in \mathbb{R}^N$ is an ``epsilon''
displacement of $w$ in $N$-dimensional space. So as a first-order
approximation:
\[
\varepsilon^{\mathsf{T}}\nabla{f}(w) \approx f(w+\varepsilon) - f(w)
\]

\subsection{Gradient descent}
For a function $f:\mathbb{R}\to\mathbb{R}$, we know $f(w')$ is a
(local) maximum or minimum if
$\frac{\text{d}f}{\text{d}w}_{w'}=0$. But what about for functions
that are $\mathbb{R}^N\to\mathbb{R}$? We require the partial
derivative along \textit{each} axis $i = 1,2,\ldots,N$ to be zero, or in other words:
\[
\left.\nabla{f}\right|_{w'} = 0
\]

The \textit{gradient descent} algorithm is an iterative approach to
minimizing or maximizing $f$. Starting from a random guess, it tries
to find a point $w_{\text{opt}}$ such that
$\nabla{f}|_{w_{\text{opt}}} = 0$ by moving \textit{one step} in a
line that's in the opposite direction to the gradient at the current
guess, see the figure below:

\begin{figure}[htbp]
  \begin{center}
    \usetikzlibrary{shadows}
\begin{tikzpicture}
%%\draw [very thin, step=1cm] (-5,-5.5) grid (5.5,4);
\node (v1) at (-4,-3) {};
\node (v11) at (-5, -2) {};
\node (v2) at (5,-2) {};
\draw [very thick, -latex](v11) -- (v2);
\node (v3) at (-4,4) {};
\draw [very thick, -latex] (v1) edge (v3);
\node (v4) at (-3,3) {};
\node (v5) at (1,-1) {};
\node (v6) at (5,3) {};
\draw plot[smooth, tension=.7] coordinates {(v4) (v5) (v6)};
\node [shape=circle] (v14) at (4,1.5) {};
\node (v15) at (4,-2.5) {$\mathbf{w_0}$};
\draw [dashed] (4,-2) -- (4,1.5);
\node [shape=circle,  minimum size=0.02cm,  circular drop shadow,  text=black,  very thick,  draw=black!55,  top color=white, bottom color=gray!80, align=center] (v16) at (4,1.5) {};
\node (v18) at (5, 2.75) {$\mathbf{\frac{\partial f}{\partial w}}$};
\draw [red,thick,->](v16) -- (v18);
\node (v19) at (3,-2.5) {$\mathbf{w_1}$};
\node [shape=circle,  minimum size=0.02cm,  circular drop shadow,  text=black,  very thick,  draw=black!55,  top color=white, bottom color=gray!80, align=center] (v20) at (3,.3) {};
\end{tikzpicture}

  \end{center}
  \caption{One step of gradient descent algorithm.}
\end{figure}

One update of the algorithm can be written as:
\[
w_{n+1} = w_{n} - \left.\nabla{f}\right|_{w_n}
\]

\section{Linear regression}
\subsection{Gradients for linear regression}

\section{Perceptron learning}

\section{Logistic regression}

\subsection{Multi-class logistic regression}

\subsection{Equalivalence between sigmoid and softmax loss functions}

\section{Support vector machines}
\subsection{Online algorithms via PEGASOS}
\end{document}
